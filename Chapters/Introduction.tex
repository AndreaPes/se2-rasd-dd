\chapter{Introduction}

\section{Purpose}
As internships become a fundamental part of academic and professional careers, providing students with easier access to opportunities is more important than ever. Traditionally, internship management has relied on fragmented systems, where students struggle to find opportunities, companies face difficulties in attracting the right candidates, and universities have limited visibility into the process.

The Student\&Company (S\&C) Platform aims to bridge these gaps. It provides a centralized system where students can explore internships, enhance their CVs, and apply with ease. Companies benefit by gaining access to tools to improve their internship postings and efficiently manage applications. Meanwhile, universities can monitor the process, mediate issues, and ensure fair practices.

This platform is designed to create a more equitable, efficient, and user-friendly experience for students, companies, and universities, ensuring a collaborative and successful internship process.

S\&C, therefore, intends to serve the following primary goals:

\begin{goalist}
    \item \textbf{Allow students to look for internships and stay updated about new opportunities}
    
        Students can explore various internship opportunities available on the platform and receive notifications about new listings
    
    \itemgb \textbf{Allow students to improve their CVs}
    
        Students can use the platform’s tools and suggestions to improve their CVs, making them more appealing to companies 
    
    \itemgc \textbf{Allow students to apply for an internship}
    
        Students can submit applications for internships directly through the platform, ensuring a seamless and efficient process 
    
    \itemgd \textbf{Allow a company to hire interns}
    
        Companies can use the platform to post internship opportunities, evaluate applications, and select the most suitable candidates for their needs 
    
    \itemge \textbf{Allow companies to improve their internship proposals}
    
        The platform provides feedback and suggestions to help companies refine their internship descriptions, making them more attractive to applicants 
   
    \itemgf \textbf{Allow users to track and monitor the status of the internship}
    
        Both students and companies can track the progress of their applications or internship postings, gaining insights into the status and next steps 
    
    \itemgg \textbf{Allow users to report feedback and complaints}
    
        The platform allows users to provide feedback or report complaints about the internship process, ensuring accountability and resolution 
    
    \itemgh \textbf{Allow universities to review and address complaints}
    
       Universities can review and address complaints submitted by students or companies and eventually end the process if necessary. This promotes fairness and transparency throughout the internship process 

\end{goalist}

\section{Scope}

The Students\&Companies (S\&C) platform is designed to facilitate the process of matching university students seeking internships with the companies offering them. The scope of the platform includes providing a centralized, user-friendly interface in which students, companies, and universities can interact to support the internship process from the beginning to the end. 

The platform supports three main users: Students, Companies, and Universities, each with specific functionalities.
\subsubsection*{Student}
Their functionalities  include:
\begin{itemize} 
    \item \textbf{CV Creation}: Upload information to generate a CV that can be improved with suggestions, making it more appealing for internships. 
    \item \textbf{Search Internship}: Browse for internship opportunities that align with their skills and preferences, filtering the options based on given criteria.
    \item \textbf{Internship Application}: Apply for internships that meet their interests.
    \item \textbf{Internship Recommendation}: Visualize internship offers that align with the student's personal interest by leveraging recommender systems.
    \item \textbf{Complaint Submission}: Submit complaints whenever issues arise during the internship process and track their resolution.
    \item \textbf{Progress Tracking}: Monitor the status and progress of ongoing applications and active internships.
\end{itemize}

\subsubsection*{Company}
Their functionalities  include:
\begin{itemize} 
    \item \textbf{Internship Posting}: Post detailed internship descriptions, including requirements, skills, and benefits. Receive system-generated suggestions to make internships more appealing to students.
    \item \textbf{Student Recommendation}: Visualize student profiles that align with the company's internship proposal by leveraging recommender systems.
    \item \textbf{Candidate Interviews}: Schedule and carry out interviews through questionnaires with accepted candidates and gather feedback.
    \item \textbf{Candidate Selection}: Select students for hiring based on the outcomes of the interview and the overall selection process.
    \item \textbf{Complaint Submission}: Submit complaints about students' performance or internship-related issues and monitor their resolution.
    \item \textbf{Internship Management}: Track and monitor the status of internships with each student.
\end{itemize}

\subsubsection*{University}
Their functionalities  include:
\begin{itemize} 
\item \textbf{Internship Monitoring}: Oversee the status and progress of university-affiliated students' internships, having access to their feedback and performance metrics.
\item \textbf{Complaint Handling}: Review and manage complaints, including those requiring intervention or an internship's potential termination.
\end{itemize}


\subsection{World Phenomena}
\begin{enumerate}[label={\textbf{[WP\arabic*]}}, leftmargin=1.52cm]
    \item Student decides to look for an internship 
    \item Company decides to open a new internship position 
\end{enumerate}

\subsection{Machine Phenomena}
\begin{enumerate}[label={\textbf{[MP\arabic*]}}, leftmargin=1.52cm]
    \item System collects statistics 
    \item System updates the status of the matchmaking
\end{enumerate}

\subsection{Shared Phenomena}
\subsubsection*{World Controlled}
\begin{enumerate}[label={\textbf{[SP\arabic*]}}, leftmargin=1.52cm]
    \item Student inserts CV informations 
    \item Company inserts new internship 
    \item Student searches for internship 
    \item Student applies for internship 
    \item User monitors the application process 
    \item Company approves application 
    \item Company writes the questionnaire 
    \item Company sends the questionnaire 
    \item Student answers the questionnaire 
    \item Company evaluates the questionnaire 
    \item Company sends evaluated questionnaire 
    \item User is notified about changes in the application process
    \item Student accepts internship position 
    \item University monitors internships 
    \item Student and company complain about the internship
    \item University handles complaints 
    \item University interrupts the internship 
    \item Student and company provide feedback and suggestions
\end{enumerate}
\subsubsection*{Machine Controlled}
\begin{enumerate}[label={\textbf{[SP\arabic*]}}, leftmargin=1.52cm]
    \setcounter{enumi}{18}
    \item Student improves their CV 
    \item Company improves their internship proposal
    \item User receives recommendations based on their preferences
    \item Student is notified when an internship that might interest them becomes available
    \item Company is notified about the availability of student CVs corresponding to its needs
    \item System provides suggestions to improve CVs and project description
    \item System asks students and companies to provide feedback and suggestions
\end{enumerate}

\section{Definitions, Acronyms, Abbreviations}

\subsection{Definitions}
\begin{itemize}
    \item \textbf{User}: A term used to refer to any of the three actors interacting with the S\&C platform: \textbf{Student}, \textbf{Company}, and \textbf{University}. It is used throughout the document when a feature or functionality applies to all three actor types or when the distinction between actor roles is not critical to the context.
\end{itemize}
\subsection{Acronyms}
\begin{itemize}
    \item \textbf{S\&C}: Students\&Companies
    \item \textbf{CV}: Curriculum Vitae
    \item \textbf{UI}: User Interface
    \item \textbf{GDPR}: General Data Protection Regulation
    \item \textbf{CCPA}: California Consumer Privacy Act
    \item \textbf{WCAG}: Web Content Accessibility Guidelines
\end{itemize}
\subsection{Abbreviations}
\begin{itemize}
    \item \textbf{[G\textit{n}]}: Goal number \textit{n}
    \item \textbf{[D\textit{n}]}: Domain assumption number \textit{n}
    \item \textbf{[UC\textit{n}]}: Use Cases number \textit{n}
\end{itemize}
\section{Revision history}
\begin{itemize}
    \item \textbf{Version 1.0}: First Release (8/12/24)
\end{itemize}
\section{Reference Documents}
For this project, we used the following documents: 
\begin{itemize}
    \item Assignment RDD AY 2024-2025
    \item Software Engineering 2 slides
\end{itemize}

\section{Document Structure}

\begin{itemize}
    \item\textbf{Section 1: Introduction} 
\end{itemize}
This section provides a summary of the application's context and purpose, highlighting its domain and the shared phenomena it observes and/or controls. Definitions, acronyms, and abbreviations used throughout the document are listed to facilitate understanding. In addition, a changelog is included for document revisions, and an overview of the document’s structure is presented, outlining the objectives of each section.

\begin{itemize}
    \item\textbf{Section 2: Overall Description} 
\end{itemize}

This section offers an overview of the structure of the application. It helps to identify the core features and potential stakeholders. The assumptions made regarding the users and their interactions with the application are detailed. To provide clarity, UML diagrams, including Class and State diagrams, and interfaces are introduced to depict the system's entities and functionalities, helping to illustrate how the application operates within its domain.

\begin{itemize}
    \item\textbf{Section 3: Specific Requirements} 
\end{itemize}

This section delves into the application’s behavior and requirements. It begins by presenting scenarios that exemplify typical interactions, followed by UML diagrams, such as Sequence and Use Case diagrams, to clarify the operational workflows. The requirements are then categorized into three key areas:

\begin{enumerate}
    \item{Functional Requirements:}
    Detailing the application's essential features and capabilities.
    \item{Non-functional Requirements:} 
    Addressing performance, reliability, and other quality aspects.
    \item{Interface Requirements:} 
    Exploring the application’s user interfaces, supported by visual mockups.
\end{enumerate}


\begin{itemize}
    \item \textbf{Section 4: Formal Analysis through Alloy}
\end{itemize}

In this section, a formal analysis of the problem is presented using the Alloy modeling tool. Requirements and domain assumptions are rigorously defined and tested for consistency within the system. This formal approach ensures that the application’s design aligns with its specified goals and supports the integrity of its implementation.


