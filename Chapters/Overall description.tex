\chapter{Overall description}

\section{Product perspective}

\subsection{Scenarios}

In the following chapter, we present a hypothetical scenario to better illustrate the phenomenon:

\begin{itemize}[leftmargin=*, label={}]
    \item \textbf{Hoenn Pokéball Innovation Company inserts an internship:}
    \begin{itemize}
        \item The Hoenn Pokéball Innovation Company, a leader in integrating AI systems into Pokéballs, seeks interns who can bring fresh perspectives to their research. Upon learning about the S\&C platform, they decide to list their internship opportunity there, hoping to attract enthusiastic candidates eager to collaborate.
    \end{itemize}

    \item \textbf{Hoenn Pokéball Innovation Company improves its internship proposal:}
    \begin{itemize}
        \item As this is the first time the Hoenn Pokéball Innovation Company is offering an internship, they are unsure how appealing their proposal might be. They follow the platform's suggestions to enhance their posting, aiming to make it more attractive to potential applicants.
    \end{itemize}

    \item \textbf{Mew applies for an internship:}
    \begin{itemize}
        \item Mew, a master’s student at Kanto University, needs an internship related to his field of study to support his thesis. After discovering the S\&C platform, he decides to use it to find opportunities. Recognizing that his CV might not stand out, he follows the app’s guidance to make it more compelling. After refining his profile, he applies to several internships that align with his interests.
    \end{itemize}

    \item \textbf{Hoenn Pokéball Innovation Company analyses the applications received:}
    \begin{itemize}
        \item After receiving multiple applications, the Hoenn Pokéball Innovation Company begins evaluating candidates. They first filter out applicants who do not meet the prerequisites. Next, they assess the level of interest shown by the remaining candidates, ultimately shortlisting those they believe would bring the greatest value to their organization and match them.
    \end{itemize}

    \item \textbf{Mew checks the status of his applications:}
    \begin{itemize}
        \item Mew monitors his applications and finds that while some companies have shown no interest, others have matched him. He also received notifications about new internship opportunities. However, he chooses to proceed with the companies that have expressed interest in his application, as they align better with his aspirations.
    \end{itemize}

    \item \textbf{The company evaluates and interviews applicants:}
    \begin{itemize}
        \item To finalize their decision, the company sends a detailed questionnaire to all shortlisted candidates. The questionnaire delves into the technical knowledge required for the role and asks personal questions about their motivations and goals. Candidates respond to the questionnaire through the platform, providing the company with deeper insights.
    \end{itemize}

    \item \textbf{Complaints handling:}
    \begin{itemize}
        \item After answering all the questions in the company’s questionnaire, Mew is thrilled to be contacted by the Hoenn Pokéball Innovation Company to discuss the terms of the internship contract. However, as time passes, the company becomes increasingly unresponsive, delaying the finalization of the agreement. Frustrated by the lack of communication, Mew raises a complaint on the S\&C platform. The Kanto University administration steps in to mediate, contacting the company to address the delays. Prompted by the intervention, the company apologizes and expedites the process, providing the necessary clarifications and finalizing the contract terms.
    \end{itemize}

    \item \textbf{Finalizing the decision:}
    \begin{itemize}
        \item With the issues resolved, Mew feels reassured and officially accepts the internship. Excited to begin, he looks forward to contributing to the company’s innovative projects while gathering valuable insights for his thesis and academic growth.
    \end{itemize}
\end{itemize}


\subsection{Domain Class Diagram}

\section{Product Functions}

The Students\&Companies (S\&C) platform provides a range of functionalities designed to support the internship matching process for students and companies, with oversight by universities. The main functionalities are described below.

\subsubsection*{User Registration} 
This functionality allows students, companies, and universities to sign up for the S\&C platform and create their profiles.
\begin{itemize}
    \item \textbf{Student Registration}: Students provide personal details, skills, experiences, and upload their CVs to create a comprehensive profile.
    \item \textbf{Company Registration}: Companies set up profiles by providing organization details, industry, location, and descriptions to establish credibility.
    \item \textbf{University Registration}: Universities create accounts to monitor and manage internship activities, with permissions to address complaints and view internship progress.
    \item \textbf{Email Verification}: All users must confirm their email addresses through a verification link sent upon registration.
\end{itemize}

\subsubsection*{Profile Management and Enhancement Suggestions} 
This functionality enables users to manage and improve their profiles to increase the likelihood of successful matches.
\begin{itemize}
    \item \textbf{Profile Updates}: Students and companies can update profile information (e.g., skills, experiences for students; internship criteria for companies) as needed.
    \item \textbf{CV and Internship Posting Enhancement}: The platform provides automatic suggestions for students on how to enhance their CVs and for companies on how to improve internship postings for greater appeal.
\end{itemize}

\subsubsection*{Internship Posting and Search} 
The platform allows companies to post internships and students to search for opportunities.
\begin{itemize}
    \item \textbf{Internship Creation}: Companies can create detailed internship listings, specifying requirements, skills, tasks, compensation, and benefits.
    \item \textbf{Internship Search and Filtering}: Students can browse and filter internships based on various criteria, such as skills required, internship duration, location, and compensation.
    \item \textbf{Internship Details View}: Students can view comprehensive details about each internship, allowing them to assess compatibility with their skills and interests.
\end{itemize}

\subsubsection*{Application and Selection Process} 
This functionality manages applications and the selection process, enabling both students and companies to track application progress.
\begin{itemize}
    \item \textbf{Application Submission}: Students can apply to internships directly from the listing, and track the status of their applications.
    \item \textbf{Candidate Shortlisting}: Companies can review applications, filter candidates based on qualifications, and shortlist those who meet their requirements.
    \item \textbf{Interview Scheduling}: Companies can set up interviews with shortlisted candidates, allowing them to assess fit further before making a final selection.
    \item \textbf{Final Selection and Offer Acceptance}: Companies send offers to selected students, who can accept or decline the offer through the platform.
\end{itemize}

\subsubsection*{Recommendation System} 
The S\&C platform provides a recommendation system to facilitate suitable matches between students and internships.
\begin{itemize}
    \item \textbf{Internship Recommendations for Students}: The platform recommends internships to students based on keyword matching and statistical analysis of skills, experiences, and internship requirements.
    \item \textbf{Candidate Recommendations for Companies}: Companies receive recommendations of students whose profiles align with internship requirements, enabling them to quickly find potential matches.
    \item \textbf{Feedback-Driven Refinements}: Both students and companies can provide feedback on recommendations, helping to improve the accuracy and relevance of future matches.
\end{itemize}

\subsubsection*{Feedback and Complaint Handling} 
This functionality supports feedback collection and complaint management to maintain quality and address issues in the internship process.
\begin{itemize}
    \item \textbf{Post-Internship Feedback}: After internships, students and companies can provide feedback, offering insights to improve future matches and platform recommendations.
    \item \textbf{Complaint Submission}: If issues arise during an internship, students and companies can submit complaints. The complaint system ensures that complaints are logged, tracked, and managed through the appropriate channels.
    \item \textbf{University Intervention}: Universities are notified of complaints and can intervene when necessary, including taking actions to resolve or terminate internships if needed.
\end{itemize}

\subsubsection*{Progress Tracking} 
This feature allows students, companies, and universities to monitor the progress of applications and active internships.
\begin{itemize}
    \item \textbf{Application Status Updates}: Students and companies can view and track the status of each application, from submission through final selection.
    \item \textbf{Internship Progress Monitoring}: Once an internship begins, students and companies can monitor ongoing activities and report any issues.
    \item \textbf{University Oversight}: Universities have access to track students’ progress in internships and monitor for educational compliance and student welfare.
\end{itemize}

\subsubsection*{Notification System} 
The notification system keeps users informed of critical events and updates.
\begin{itemize}
    \item \textbf{Application and Selection Notifications}: Students are notified of changes in application status, such as interview invitations or selection decisions.
    \item \textbf{Recommendation Alerts}: Students and companies receive alerts for recommended internships or candidate profiles that match their preferences.
    \item \textbf{Complaint and Resolution Updates}: Notifications are sent to relevant parties when complaints are filed or resolved, ensuring transparent communication.
    \item \textbf{Reminder and Follow-Up Alerts}: The system sends reminders for upcoming interviews, deadlines, and other key events to keep users engaged and informed.
\end{itemize}


In summary, the S\&C platform provides a structured, streamlined approach to managing the internship application process from start to finish, including registration, profile management, recommendation, application handling, feedback collection, and progress tracking. This section outlines the essential product functions, organized by feature, providing a clear view of the platform’s primary capabilities.



\section{User characteristics}

\subsubsection*{Student}
The student is a person who studies at a university and is interested in applying for and participating in internships offered by companies. For this reason, they register on the S\&C platform, where they can find internships either through an active search or via recommendations. After applying for an internship, they must pass an interview with the company offering the position. Once the internship begins, the student can communicate any issues or provide updates to both the university and the company using a chat or form feature.
\subsubsection*{Company}
The university is the institution where the student studies. It can review and manage complaints submitted by other users and may decide to interrupt the internship based on these considerations.

\subsubsection*{University}
The company is interested in hiring students through an internship program. It can post work opportunities on the platform, where students can view them. Afterward, the company can receive feedback from the system to support the recruitment process. Once the student is selected, the company may use the platform to send feedback and file complaints to other users during the internship period.


\section{Assumptions, dependencies and constraints}

\subsubsection*{Domain Assumptions}
The following assumptions are properties and conditions that the system (CKB platform) will take for granted. These assumptions are necessary to ensure correct platform behavior and to achieve the intended goals, and they must be verified to maintain proper functionality.
\begin{domainlist}
    \item User must have an initial CV
    \itemdb User must consent to personal data extraction and usage
    \itemdc User must consent to receiving information
    \itemdd Users must provide correct personal information at the moment of registration
    \itemde User must consent to personal data extraction and usage
    \itemdf Student must be enrolled in a university
    \itemdg Company must provide correct and clear information about the internships

\end{domainlist}