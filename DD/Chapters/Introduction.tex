\chapter{Introduction}

\section{Purpose}
Internships have become a cornerstone of academic and professional development, making it increasingly vital to provide students with streamlined access to opportunities. However, traditional internship management often relies on disjointed systems, where students struggle to discover suitable opportunities, companies face challenges in attracting qualified candidates, and universities have limited oversight of the process.

The Student\&Company (S\&C) Platform aims to address these challenges by offering a centralized system that bridges these gaps. Through this platform:

\begin{itemize}
    \item \textbf{Students} can easily explore internship opportunities, enhance their CVs, and apply efficiently.
    \item \textbf{Companies} gain access to advanced tools for improving internship postings and managing applications effectively.
    \item \textbf{Universities} can monitor the entire process, mediate issues, and ensure fair practices across all parties involved.
\end{itemize}

This document provides a comprehensive overview of the platform's functionalities and constraints. It is intended for:

\begin{itemize}
    \item \textbf{Developers:} To serve as a clear reference for the implemented requirements and as a basis for agreement between the customer and contractors.
    \item \textbf{The Customer:} To deliver an unambiguous description of the system’s capabilities, enabling validation of the requirements and verification that the system meets expectations.
\end{itemize}


\section{Scope}

The Students\&Companies (S\&C) Platform is designed to streamline the process of connecting university students seeking internships with companies offering them. By providing a centralized and user-friendly interface, the platform fosters effortless interactions between students, companies, and universities, supporting the entire internship lifecycle from start to finish.

The platform caters to three primary user groups—Students, Companies, and Universities—each equipped with specific functionalities tailored to their roles:

Students can create and enhance their CVs, browse and apply for internships that align with their skills and interests, and receive personalized recommendations through a recommender system. They can also submit complaints to address any issues, track their applications and internships, and monitor progress throughout the process.

Companies can post detailed internship opportunities, receive suggestions to improve postings, and access profiles of recommended candidates. They can schedule and conduct interviews, select candidates, and manage active internships efficiently. In addition, companies can address performance or internship-related issues through the complaint submission and resolution system.

Universities play a supervisory role by monitoring the progress and feedback of affiliated students’ internships. They can handle complaints, intervene when necessary, and ensure internships meet academic and ethical standards.

\section{ Definition, Acronyms, Abbreviations}

\begin{longtable}{|l|l|}
    \hline
    \textbf{Acronym} & \textbf{Definition} \\
    \hline
    DD & Design Document \\
    \hline
    RASD & Requirements Analysis \& Specification Document \\
    \hline
    ST & Student \\
    \hline
    UNI & University \\
    \hline
    STG & Student Group \\
    \hline
    S\&C & Student\&Company \\
    \hline
    UC & Use Case \\
    \hline
    UI & User Interface\\
    \hline
    User & All Students, Companies and Universities \\
    \hline
    API & Application Programming Interface \\
    \hline
    RX & Requirement X \\
    \hline
    CMP & Component \\
    \hline 
    UML & Unified Modelling Language \\
    \hline
    DB & Database \\
    \hline
    DBMS & Database Management System \\
    \hline
     REST & Representational State Transfer \\
     \hline
     MCV & Model View Control \\
    \hline
\end{longtable}

\section{ Revision History}

 The following are the revision steps made by the team during the DD development:

 \begin{itemize}
     \item \textbf{Version 1.0} - 07/01/2025
 \end{itemize}

\section{ Reference Documents}

\begin{itemize}
    \item Specification of RASD and DD assignment
    \item  Slides of the course of Software Engineering 2
\end{itemize}

\section{Document Structure}

The document is organized into seven sections, as outlined below:

\begin{itemize}
    \item \textbf{Introduction:}  
    This section highlights the importance of the Design Document and provides clear definitions and explanations of acronyms and abbreviations. It also revisits the scope of the Student\&Company system, setting the stage for the subsequent sections.

    \item \textbf{Architectural Design:}  
    This section presents the primary components of the system and their interrelationships. It emphasizes key design decisions, architectural styles, patterns, and paradigms adopted for the system.

    \item \textbf{User Interface Design:}  
    The third section describes the system's user interface, including mockups and detailed explanations of the main pages, illustrating how users interact with the platform.

    \item \textbf{Requirements Traceability:}  
    This section outlines the system requirements and demonstrates how the design decisions fulfill them, ensuring a clear traceability of requirements throughout the design process.

    \item \textbf{Implementation, Integration, and Test Plan:}  
    This part provides an overview of the implementation of the system's components, describes their integration, and outlines a comprehensive plan for testing them to ensure functionality and reliability.

    \item \textbf{Effort Spent:}  
    The sixth section documents the number of hours each team member contributed to developing this Design Document, offering transparency regarding the effort distribution.

    \item \textbf{References:}  
    The final section lists all the documents and resources used in drafting this Design Document, ensuring proper attribution and traceability of information sources.
\end{itemize}